%! Author = Philipp Emmenegger
%! Date = 10/06/2021

\section{Multithreading Grundlagen}
\subsection{Multi-Thread Programmierung}
\subsubsection{JVM Thread Modell}
Java ist ein Single Process System. 
JVM ist ein Prozess im Bsys.
\textbf{Main-Thread} wird beim Aufstarten der JVM anhand \textit{main()} Methode erzeugt.
JVM läuft, solange Threads laufen (Ausnahme Daemon Threads).
\begin{lstlisting}
var myThread = new Thread(() -> { /* ... */ });
myThread.start();
\end{lstlisting}
Thread wird erst bei \textit{start()} erzeugt. Führt \textit{run()}-Methode des Runnable Interface aus.
Thread endet beim Verlassen von \textit{run()}.\\
\textbf{Nicht-Determinismus:} Threads laufen ohne Vorkehrungen beliebig verzahnt oder parallel.
\subsubsection{Explizite Runnable-Implementation:}
\begin{lstlisting}
class SimpleLogic implements Runnable {
    @Override
    public void run() { /* ... */ }
}
var myThread = new Thread(new SimpleLogic()).start();
\end{lstlisting}

\subsubsection{Sub-Klasse von Thread}
\begin{lstlisting}
class SimpleThread extends Thread {
    @Override 
    public void run() { /* ... */ }
}
var myThread = new SimpleThread().start();
\end{lstlisting}

\subsubsection{Thread Join}
Warten auf Beendigung eines Threads. \textit{t2.join()} blockiert, solange t2 läuft.

\subsubsection{Thread Passivierung}
\textbf{Thraed.sleep(ms):} Laufender Thread geht in Wartezustand, dann ready.
\textbf{Thread.yield():} Gibt Prozesor frei, direkt ready.