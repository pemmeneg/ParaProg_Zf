\section{Thread Pool}
\textbf{Work-Stealing:} Vorteil: Effizienz duch weniger Contention.
Nachteil: Fairnessprobleme bei unausgeglichenen Verteilungen.
\subsection{Vorteile}
\textbf{Beschränkte Anzahl von Threads:} Viele Threads verlangsamen das System oder überschreiten Speicherlimit. 
\textbf{Recycling der Threads:} Spare Thread-Erzeugung und Freigabe.
\textbf{Höhere Abstraktion:} Trenne Task-Beschreibung von Task-Ausführung.
\textbf{Anzahl Threads pro System konfigurierbar:} \#Worker Threads = \#Prozessoren + \#I/O-Aufrufe 

\subsection{Einschränkung}
Daemon: Laufen nicht garantiert zu Ende bei Fire \& Forget.
Tasks dürfen nicht aufeinander Warten, sonst Deadlock.
\textbf{Run to Completion:} Task muss zu Ende laufen, bevor Worker Thread anderen Task ausführen kann.
\textbf{Ausnahme:} geschachtelte Tasks.

\subsection{Java Fork-Join-Pool}
\begin{lstlisting}
var threadPool = new ForkJoinPool();
Future<Integer> future = threadPool.submit(() -> {  });
Int result = future.get(); // Blockiert bis Task beendet
\end{lstlisting}

\subsubsection{Rekursive Task Erstellung}
\textbf{Anzahl Tasks:} Java: Arraylänge, THRESHOLD.
.NET: \# Freien Worker Threads falls \textless Arraylänge
\begin{lstlisting}
class PairSwapTask extends RecursiveTask<Integer> {
    private static final int THRESHOLD = 1000;
    private final int[] array;
    private final int from, to;
    public PairSwapTask(int[] array, int from, int to) {
        this.array = array; this.from = from, this.to = to;
    }
    @Override
    protected Integer compute() {
        if(to - from <= THRESHOLD) {
            for(int i = from; i < to; i++) {
                int offset = 2 * i;
                int temp = array[i];
                array[offset] = array[offset + 1];
                array[offset + 1] = temp;
            }
        } else {
            int middle = (from + to) / 2;
            // l = new, r = new, fork(), fork(), return join() + join();
            invokeAll(
                new PairSwapTask(array, from, middle);
                new PariSwapTask(array, middle, to);
            );
        } } }
// Ausfuehrung 
var threadPool = new ForkJoinPool();
threadPool.invoke(new PairSwapTask(array, 0, array.length / 2));
\end{lstlisting}

\subsubsection{Fork Join Pool Internals}
\textbf{Automatischer Parallelitätsgrad:} Default: \#Worker Threads = \#Prozessoren
Dynamisches Hinzufügen / Wegnehmen von Threads
\textbf{Common Pool:} Verhindert Engpässe durch zu viele Thread Pools.


\subsection{Asynchrone Programmierung}
\textbf{Unnötige Synchronität:} Langlaufende Rechnungen, I/O Aufrufe.\\ 
\textbf{Asynchroner Aufruf:} Aufrufer soll während der Operation weitermachen.

\begin{lstlisting}
// runAsync() falls kein Rueckgabewert
CompletableFuture<long> future = CompletableFuture.supplyAsync(()->{});
// other work 
process(future.get());
\end{lstlisting}

\subsubsection{Ende des async Aufrufs}
\textbf{Caller-zentrisch (Pull):} Caller warted auf Task-Ende und holt sich Resultat, Future abfragen.
\textbf{Callee-zentrisch (Push):} Async Operation informiert direkt über Resultat. Completion Callback.

\subsubsection{Continuation}
Folgeaufgabe an asynchrone Aufgabe anhängen.
\begin{lstlisting}
// thenApply() fuer Continuation mit Rueckgabe
future.thenAccept(res -> System.out.println(res));
\end{lstlisting}
\textbf{Ausführung:} durch beliebigen Thread, durch Initiator, wenn Future bereits Resultat hat.
\textbf{Asynchrone Continuations:} \textit{thenAcceptAsync()} bzw. \textit{thenApplyAsync()}.

\subsubsection{Multi-Continuation}
\begin{lstlisting}
// .allOf(futures).thenRun(() -> syso());
CompletableFuture.allOf(f1, f2).thenAcceptAsync(() -> { }));
CompletableFuture.any(f1, f2).thenAcceptAsync(() -> { }));
\end{lstlisting}

\subsubsection{Fire and Forget}
Task starten, ohne das Ende abzuwarten. Submitter ruft kein \textit{get()} oder \textit{join()} auf.
\begin{lstlisting}
CompletableFuture.runAsync(() -> { });
\end{lstlisting}
\textbf{Daemon Workers:} Workers Threads in Fork-Join-Pools sind Daemon.
Anwendung kann vor Task-Ende stoppen.\\ 
\textbf{Ingorierte Exceptions:} Exceptions in Fire \& Forget Task werden ignoriert.
