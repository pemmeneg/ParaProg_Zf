\section{Concurrency in Python}
\textbf{GIL:} Global Interpreter Lock. 
Nur ein Thread kann Python Byte-Code ausführen.
Kein Speedup für CPU-Bount Operationen möglich. Data Races durch Reordering dennoch möglich, Visibility nicht garantiert.
Kein definiertes Memory Model.
\begin{lstlisting}[language=python]
from threading import Thread
from multiprocessing import Process
if __name__ == '__main__':
    t = Thread( # p = Process
        target=fibonacci,
        args=(10,)
    ).start();
    t.join();
# Shared Memory
res = Value('i', -1, lock=False); // Typ, InitialWert, lock
\end{lstlisting}

\subsection{Pools}
\textbf{Threads:} \textit{concurrent.futures.thread.ThreadPoolExecutor}
\textbf{Prozesse:} \textit{concurrent.futures.process.ProcessPoolExecutor} / \textit{multiprocessing.Pool}
\begin{lstlisting}[language=python]
if __name__ == '__main__':
    with ProcessPoolExecutor() as pool:
        future = pool.submit(fib, 10) # or: map(fib, [1,2,3])
        print(future.res()); # Blockiert bis Task-Ende
\end{lstlisting}

\subsection{Asnychrone Programmierung (asyncio)}
\textbf{Coroutine-Functions:} werden erst beim \textit{await} ausgeführt.
Keine parallele Ausführung. Ausnahme: Coroutine wird als Task verpackt.
\begin{lstlisting}[language=python]
async def sub_routine(n):
    await asyncio.sleep(n);
if __name__ == 'main__':
    asyncio.run(sub_routine(1))
\end{lstlisting}


\section{JavaScript Concurrency}
Grundsätzlich Single-Threaded mit einem Event-Loop.
Kein Schutz vor Race-Condition.
\begin{lstlisting}
function delay(ms) {
    let promise = new Promise((resolve, reject) => {
    setTimeout(() => resolve(), ms); }); }
async function countTo(n) {
    for (let = 1; i <= n; i++) {
        await delay(1000); } }
\end{lstlisting}
\subsection{Web-Worker}
Entsprechen einem Thread im Browser. Datenaustausch primär über Messaging.
Werden mit Quelldatei gestartet. Langlebiger Prozess mit eigenem Event-Loop.
\begin{lstlisting}
onmessage = event => { // Worker definieren
    const n = event.data; const res = fib(n);
    postMessage(res); }
const worker = new Worker('w.js'); // Worker verwenden
worker.onmessage = event => {
    console.log(event.data); }
worker.postMessage(42);
\end{lstlisting}