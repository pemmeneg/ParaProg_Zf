\section{Memory Models}
\textbf{Lock-Freie Programmierung:} Korrekte nebenläufige Interaktionen ohne Locks.
Garantien des Speichermodels nutzen.
\subsection{Ursachen für Probleme}
\textbf{Weak Consistency:} Speicherzugriffe werden in verschiedenen Reihenfolgen aus verschiedenen Threads gesehen.
Ausnahme: Synchronisationen/Speicherbarrieren
\textbf{Optimierungen:} Compiler, Laufzeitsystem und CPUs, Instruktionen werden umgeordnet/wegoptimiert.

\subsection{Java Memory Model}
\subsubsection{Atomicity}
\begin{lstlisting}
var myInt = new AtomicInteger(1);
var myObj = new AtomicReference(null);
myInt.updateAndGet(x -> x * x);
var temp = myInt.getAndSet(2); // .setAndGet(2)
var temp1 = myInt.incrementAndGet(); // .getAndDecrement();
\end{lstlisting}
Einzelnes Lesen / Schreiben ist atomar für: Primitive Datentypen, Obj-Referenzen, long und double nur mit \textit{volatile} Keyword.\\ 
\textbf{Unteilbarkeit $\neq$ Sichtbarkeit:} Nach Write sieht anderer Thread vlt. noch alten Wert.

\subsubsection{Visibility}
\textbf{Garantierte Sichtbarkeit zwischen Threads:}
Locks Release \& Acquire, Volatile Variable, Thread/Task-Start und Join, Initialisierung von final Variablen.

\subsubsection{Ordering}
\textbf{Program Order:} 'as-if-serial', Sequentielles Verhalten jedes Threads bleibt erhalten. (Andere Threads dürfen es anders sehen)
\textbf{Synchronization Order (Total Order):} Synchronizationsbefehle werden zueinander nie umgeordnet.
\textbf{Happens-Before Relation (Partial Order):} Alles andere kann umgeordnet werden, ausser es gibt garantierte Sichtbarkeit unter den Threads.

\subsection{Java Synchronization Order}
\begin{lstlisting}
// Keine Umordnung in Java, weil alles volatile
volatile boolean a = false, b = false;
a = true; while(!b) { } // Thread 1 
b = true; while(!a) { } // Thread 2 
// Nicht korrekt, da nicht atomar 
private volatile boolean locked = false;
public void aqcuire() {
    while(locked) { }
    locked = true;
}
// Spin-Lock mit atomarer Operation
private AtomicBoolean locked = new AtomicBoolean(false);
public void acquire() {
    while (locked.getAndSet(true)) { }
}
public void release() { locked.set(false); }

public void aqcuire() {
    boolean success = false;
    do {
        boolean old = locked.get();
        if(old != true) {
            success = locked.compareAndSet(old,true);
        }
    } while(!success);
}
public void release() {
    do {
        boolean old = locked.get();
    } while(locked.compareAndSet(old,false));
}
// Atomares Compare and Set, setzt Update falls Wert gleich expect
boolean compareAndSet(boolean expect, boolean upadte); 
// Optimistische Synchronization
do {
    oldValue = var.get(); newValue = calcChange(oldValue);
} while (!var.compareAndSet(oldValue, newValue));
\end{lstlisting}

\subsection{.NET Memory Model}
\textbf{Unterschied zu Java:} Atomicity: long/double auch ohne volatile atomar.
Visibility: Nicht definiert, implizit durch Ordering.
Ordering: nur Half und Full Fences.\\ 
\textbf{Atomare Instruktionen:} \textit{Interlocked} Klasse

\subsubsection{Volatile Half Fences}
\textbf{Volatile Write:} Vorangehende Zugriffe bleiben davor. (Release Semantik)
\textbf{Volatile Read:} Nachfolgende Zugriffe bleiben danach. (Acquire Semantik)

\subsubsection{Full Fence: Memory Berrier}
\begin{lstlisting}[language=csh]
Thread.MemoryBarrier(); // Verbietet Umordnung in beide Richtungen
\end{lstlisting}