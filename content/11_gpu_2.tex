\subsection{Speicherstufen}
\textbf{Shared Memory:} Per SM.
Schnell ca 4 Zyklen.
Nur zwischen Threads innerhalb Block sichtbar.
Paar KB.
\textit{\_\_shared\_\_ float x;}
\textbf{Global Memory:}
Main Memory in GPU.
Langsam, ca. 400-600 Zyklen.
Allen Threads sichtbar.
Mehrere GB.
\textit{cudaMalloc()}

\subsection{Block Barriere}
\begin{lstlisting}
__syncThreads();
\end{lstlisting}
In if/else nur falls für alle Threads eines Blocks.

\subsection{Warp}
Block wird intern in Warps zerlegt (je 32 Threads).
\textbf{Ausführung:} SIMD.
SM kann alle Warps eines Blocks beherbergen. Nur wenige laufen echt parallel (1 bis 24).

\subsection{Divergenz}
Unterschiedliche Verzweigungen im selben Warp.
\textit{if/switch/while/do/for}.
SM führt Intruktionen der einen Verzweigung durch, dann Instruktionen einer anderen Verzweigung.
\textbf{Performance Problem}
Schlechter Fall: Divergenz innerhlab derselben Warp.
Guter Fall: Gleiche Verzweigung im Warp.

\subsection{Memory Coalescing}
Zugriffsmuster für Performance optimieren.
\textbf{Burst:}  Falls Threads auf 32-Byte-Bereiche zugreifen.
Sonst teure Einzel-Zugriffe. (je 400 Zyklen pro Global Memory)\\ 
\textbf{Tiles einlesen}, \textbf{Multi-Dim} Spalte auf x-Achse

\subsection{Shared Memory Optimization}
Falls Threads auf mehrere Elemente zugreifen, im SM Cachen.

\subsection{Example}
\subsubsection{1-Dimensional}
\begin{lstlisting}
__global__ void pairwise_sum(int* array, int length) {
    int i = blockIdx.x * blockDim.x + threadIdx.x;
    int offset = 2 * i;
    if(offset + 1 < length) {
        array[offset] += array[offset + 1];
        array[offset + 1] = 0;
    }
}
// Aufruf
int blockSize = 1024;
int gridSize = (len / 2 + blockSize - 1) / blockSize;
pairwise_sum<<<gridSize, blockSize>>>(d_a, len);
\end{lstlisting}

\subsubsection{2-Dimensional}
\begin{lstlisting}
__global__ void transpose(int* matrix, int rows, int cols, int* result) {
    int row = blockIdx.x * blockDim.x + threadIdx.x;
    int col = blockIdx.y * blockDim.y + threadIdx.y;
    if(row < rows && col < cols) {
        result[col * rows + row] = matrix[row * cols + col];
    }
}
// Aufruf
dim3 blockSize(32,32);
dim3 gridSize((N + 31) / 32, (M + 31) / 32);
transpose<<<gridSize, blockSize>>>(matrix, N, M, result);
\end{lstlisting}