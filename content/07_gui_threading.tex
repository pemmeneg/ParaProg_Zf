\section{GUI and Threading}
GUI Frameworks erlauben nur Single-Threading. \\ 
\textbf{UI Thread:} Loop zur Ausführung der Ereignisse aus einer Queue.
\subsection{UI Thread Confinement}
Wiso basieren GUI-Frameworks auf Single-Thread Modell?\\
\textbf{Synchronisationskosten:} Locking in allen Komponenten und Methoden relativ teuer.
\textbf{Deadlock-Risiko:} Bei zyklischen geschachtelten Aufrufen (z.B. MVC).

\subsection{UI Thread Model - Einschränkungen}
Keine lange Operationen in UI Events, blockiert sonst das UI.
Kein Zugriff auf UI-Elemente durch fremde Threads, sonst Race Condition. 
UI Operationen müssen als Events in die UI Event Queue eingereiht werden.

\subsection{Swing: Dispatching an UI Thread}
Komponentenzugriffe an UI Thread delegieren.
\begin{lstlisting}
// Benutzung der Klasse SwingUtilities
static void invokeLater(Runnable doRun); // Async
static void invokeAndWait(Runnable doRun); // Synchron 
// Example
button.addActionListener(event-> {
    new Thread(() -> {
        var text = readHugeFile();
        SwingUtilities.invokeLater(() -> 
                { textArea.setText(text); }); }).start(); });
\end{lstlisting}

\subsubsection{Swing Background Worker}
Hiflsklasse für Hintergrund Arbeiten.
Zeitaufwendige Operationen als Task in Thread Pool \textit{doInBackground()}.
UI-Zugriffe durch EventDispatchThread \textit{done()}.
\begin{lstlisting}
// <return, zwischen Res>
class BgCalc extends SwingWorker<Integer, Void> {
    @Override
    public Integer doInBackground() { return longCalc(); }
    @Override
    protected void done() {
        try {
            inst res = get(); label.setText(res);
        } catch (InterruptedException | ExecutionException e) 
        { } } }
\end{lstlisting}

\subsection{Android Async Task}
\begin{lstlisting}
// <input, zwischen Res, return>
class BgCalc extends AsyncTask<Void, Void, Integer> {
    @Override
    public Integer doInBackground(Void... input) {
        return longCalc(); }
    @Override
    public void onPostExecute(Integer res) {
        view.setText(res); } }
\end{lstlisting}

\subsection{.NET UI Threading Modell}
Gleiches Prinzip wie Java.
\textbf{UI Thread:} Aufrufer von \textit{Application.Run()}.\\ 
\textbf{UI Event Dispatching:}
WPF: \textit{control.Dispatcher.InvokeAsync(action)}\\ 
WinForm: \textit{control.BeginInvoke(delegate)}

\subsubsection{Async / Await}
Async Methode läuft teilweise synchron, teilweise asynchron.
Aufrufer führt Methode solange synchron aus bis ein \textit{await} anliegt.
Compiler zerlegt Methode in Abschnitte. Abschnitt nach Await läuft später nach Task-Ende (Continuation).
Methode läuft synchron bis \textit{await}, springt dann zurück zum Aufrufer.\\ 
\textbf{Verschiedene Ausführungen:} Fall 1: Aufrufer ist normaler Thread, Abschnitt wird durch TPL Worker-Thread ausgeführt.
Fall 2: Aufrufer ist UI-Thread, Abschnitt wird als Event vom UI-Thread ausgeführt.
\begin{lstlisting}[language=csh]
public async Task<int> LongOperationAsync() { }
Task<int> task = LongOperationAsync(); /* ... */
int res = await task; // Warte auf Beendigung
\end{lstlisting}
\textbf{Async Rückgabetypen:}
\textit{void}: fire-and-forget.
\textit{Task}: Kein Rückgabetyp erlaubt warten.
\textit{Task\textless T\textgreater} Rückgabetyp T.
\begin{lstlisting}[language=csh]
async Task<string> ConcatAsync(string url1, string url2) {
    HttpClient client = new HttpClient();
    Task<string> d1 = client.GetStringAsync(url1);
    Task<string> d2 = client.GetStringAsync(url2);
    string site1 = await d1; string site2 = await d2;
    return site1 + site2;
}
\end{lstlisting}